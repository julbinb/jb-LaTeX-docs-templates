%% ============================================================
\section{Вступление, которое можно и не читать}
%% ============================================================

Наверное, у каждого из вас, ребята, есть своя любимая игрушка. А может быть, даже две или пять.

У меня~\cite{1966:Uspenski:Crocodile-Gena}, например, когда я был маленьким, было три любимых игрушки: громадный резиновый крокодил по имени Гена, маленькая пластмассовая кукла Галя и неуклюжий плюшевый зверёк со странным названием — Чебурашка.

Чебурашку сделали на игрушечной фабрике, но сделали так плохо, что невозможно было сказать, кто же он такой: заяц, собака, кошка или вообще австралийский кенгуру? Глаза у него были большие и жёлтые, как у филина, голова круглая, заячья, а хвост коротенький и пушистый, такой, какой бывает обычно у маленьких медвежат.

Мои родители утверждали, что Чебурашка — это неизвестный науке зверь, который водится в жарких тропических лесах.

Сначала я очень боялся этого неизвестного науке Чебурашку и даже не хотел оставаться с ним в одной комнате. Но постепенно я привык к его странной внешности, подружился с ним и стал любить его не меньше, чем резинового крокодила Гену и пластмассовую куклу Галю.

С тех пор прошло очень много времени, но я всё равно помню своих маленьких друзей и вот написал о них целую книгу.

Разумеется, в книге они будут живые, а не игрушечные.

%% ============================================================
\section{Глава первая}
%% ============================================================

В одном густом тропическом лесу жил да был очень забавный зверёк. Звали его Чебурашка. Вернее, сначала его никак не звали, пока он жил в своём тропическом лесу. А назвали его Чебурашкой потом, когда он из леса уехал и встретился с людьми. Ведь это же люди дают зверям имена. Это они сказали слону, что он слон, жирафу — что он жираф, а зайцу — что он заяц.

Но слон, если бы подумал, мог бы догадаться, что он слон. Ведь у него же очень простое имя! А каково зверю с таким сложным именем, как гиппопотам? Поди догадайся, что ты не ги-потам, не по-потам, а именно гип-по-по-там.

Так вот и наш зверёк; он никогда не задумывался над тем, как его зовут, а просто жил себе да жил в далёком тропическом лесу.

Однажды он проснулся утром рано, заложил лапы за спину и отправился немного погулять и подышать свежим воздухом.

Гулял он себе, гулял и вдруг около большого фруктового сада увидел несколько ящиков с апельсинами. Не долго думая, Чебурашка забрался в один из них и стал завтракать. Он съел целых два апельсина и так объелся, что ему трудно стало передвигаться. Поэтому он прямо на фруктах и улёгся спать.

Спал Чебурашка крепко, он, конечно, не слышал, как подошли рабочие и заколотили все ящики.

После этого апельсины вместе с Чебурашкой погрузили на корабль и отправили в далёкое путешествие.

Ящики долго плавали по морям и океанам и в конце концов оказались во фруктовом магазине очень большого города. Когда их открыли, в одном апельсинов почти не было, а был только толстый-претолстый Чебурашка.

Продавцы вытащили Чебурашку из его каюты и посадили на стол. Но Чебурашка не мог сидеть на столе: он слишком много времени провёл в ящике, и у него затекли лапы. Он сидел, сидел, смотрел по сторонам, а потом взял да и чебурахнулся со стола на стул. Но и на стуле он долго не усидел — чебурахнулся снова. На пол.

— Фу-ты, Чебурашка какой! — сказал про него директор магазина. — Совсем не может сидеть на месте!

Так наш зверёк и узнал, что его имя — Чебурашка.

— Но как же мне с тобой поступить? — спросил директор. — Не продавать же тебя вместо апельсинов?

— Не знаю, — ответил Чебурашка. — Как хотите, так и поступайте.

Директору пришлось взять Чебурашку под мышку и отнести его в главный городской зоопарк.

Но в зоопарк Чебурашку не приняли. Во-первых, зоопарк был переполнен. А во-вторых, Чебурашка оказался совершенно неизвестным науке зверем. Никто не знал, куда же его поместить: то ли к зайцам, то ли к тиграм, то ли вообще к морским черепахам.

Тогда директор снова взял Чебурашку под мышку и пошёл к своему дальнему родственнику, также директору магазина. В этом магазине продавали уценённые товары.

— Ну что же, — сказал директор номер два, — мне нравится этот зверь. Он похож на бракованную игрушку! Я возьму его к себе на работу. Пойдёшь ко мне?

— Пойду, — ответил Чебурашка. — А что мне надо делать?

— Надо будет стоять в витрине и привлекать внимание прохожих. Понятно?

— Понятно, — сказал зверёк. — А где я буду жить?

— Жить?.. Да хотя бы вот здесь! — Директор показал Чебурашке старую телефонную будку, стоявшую у входа в магазин. — Это и будет твой дом!

Так вот и остался Чебурашка работать в этом большом магазине и жить в этом маленьком домике. Безусловно, этот дом был не самый лучший в городе. Но зато под рукой у Чебурашки всегда находился телефон-автомат, и он мог звонить кому хочешь, прямо не выходя из собственного дома.

Правда, пока ему некому было звонить, но это его нисколько не огорчало.

%% ============================================================
\section{Глава вторая}
%% ============================================================

В том городе, где оказался Чебурашка, жил да был крокодил по имени Гена. Каждое утро он просыпался в своей маленькой квартире, умывался, завтракал и отправлялся на работу в зоопарк. А работал он в зоопарке… крокодилом.

Придя на место, он раздевался, вешал на гвоздик костюм, шляпу и тросточку и ложился на солнышке у бассейна. На его клетке висела табличка с надписью:

Африканский крокодил Гена.

Возраст пятьдесят лет.

[Крокодил Гена и его друзья (повесть)]

Кормить и гладить разрешается.

Когда кончался рабочий день, Гена тщательно одевался и шагал домой, в свою маленькую квартиру. Дома он читал газеты, курил трубку и весь вечер играл сам с собой в крестики-нолики.

Однажды, когда он проигран сам себе сорок партий подряд, ему стало очень и очень грустно.

«А почему я всё время один? — подумал он. — Мне надо обязательно завести себе друзей».

И, взяв карандаш, он написал такое объявление:

МОЛОДОЙ КРАКОДИЛ ПЯТИДЕСЯТИ ЛЕТ

ХОЧЕТ ЗАВИСТИ СЕБЕ ДРУЗЕЙ.

СПРЕДЛОЖЕНИЯМИ ОБРАЩАТЬСЯ ПО АДРЕСУ:

БОЛЬШАЯ ПИРОЖНАЯ УЛИЦА, ДОМ 15, КОРПУС Ы.

ЗВОНИТЬ ТРИ С ПОЛОВИНОЙ РАЗА.

